<h2> Learn how to type mathematical equations using LaTeX </h2>
<p> LaTeX is the standard language used by mathematicians, physicists, etc. for
typesetting mathematical expressions.
This application uses <a href="https://docs.mathjax.org/en/latest/mathjax.html">MathJax</a>,
a Javascript-based engine, to process LaTeX code and render it in the way you would see it in a textbook. </p>
<h2> Delimiting your LaTeX code </h2>
<p> You must put special delimiters at the beginning and end of your LaTeX code,
in order for MathJax to recognize it and process it. If you want the equation
to appear inline with the surrounding text, put <code>\(</code> before and <code>\)</code> after
it. If you want the equation to stand alone in a separate "paragraph", centered
horizontally on the screen as in most textbooks, put <code>$$</code> before and
after it. </p>
<p> Here's an example, first with the delimiters disabled so you can see the raw
code: </p>
A quadratic equation is of the form <code>\(ax^2 + bx + c = 0\)</code>. To
solve for x, use the quadratic formula <code>$$x = \frac {-b \pm \sqrt{b^2 -4ac}}
{2a},$$</code> substituting the values of the coefficients a, b and c. 

 And now with the delimiters enabled so as to render the equations: 
A quadratic equation is of the form \(ax^2 + bx + c = 0\). To
solve for x, use the quadratic formula $$x = \frac {-b \pm \sqrt{b^2 -4ac}}
{2a},$$ substituting the values of the coefficients a, b and c. 
<p> Hereafter, I will omit the delimiters in code examples, to save space. </p>
<h2> Spaces </h2>
<p> Most blank spaces and end-of-line characters in your raw code are ignored.
Spaces will usually be placed in the rendered equation according to standard
mathematical conventions. For example, <code>a b + c / d</code> produces \(a b + c / d\)  
I'll try to note exceptions when they arise. </p>
<p> If you need to insert a visible space, the simplest way is usually to use a
<code>~</code> (tilde). You can also use <code>\text{ }</code>. There are examples
of both in the <b>Plain text</b> section below. </p>
<h2> Superscripts and subscripts </h2>
<p> The symbols <code>^</code> and <code>_</code> are the commands for
superscripts (e.g. exponents) and subscripts, respectively. If a superscript
or subscript has more than one character, enclose them in braces
<code>{ }</code>. This is a general rule for LaTeX commands, by the way:
if they are to act on multiple characters, enclose those characters in
braces. </p>
<p> You can have both subscripts and superscripts on the same symbol. Normally
they appear one on top of the other. If you want them offset as in tensor
notation, attach some of them to a pair of empty braces, as in the example for
R below. </p>
<p> If a subscript or superscript is a label that should be displayed in upright
Roman text, not italic, enclose it in <code>\rm{ }</code>. </p>
  \(x^2, y_k, e^{ikx}\) <code> x^2, y_k, e^{ikx} </code> 
 \(x^2_k, R^{ab}{}_{cd}\) <code> x^2_k, R^{ab}{}_{cd} </code> 
 \(x_{old}, x_\rm{old}\) <code>x_{old}, x_\rm{old} </code> 

<h2> Roots and fractions </h2>
<p> You can combine operations by using braces to nest them,
one inside another. See the last example below. </p>
 \( \sqrt x, \sqrt[3] x, \sqrt {b^2 - 4ac}\) <code> \sqrt x, \sqrt[3] x, \sqrt {b^2 - 4ac} </code>
 \(\frac a b, \frac {ab} {c + d}\) <code> \frac a b, \frac {ab} {c + d} </code>
\(\frac 1 {\sqrt {1 - \frac {v^2} {c^2}}}\) <code> \frac 1 {\sqrt {1 - \frac {v^2} {c^2}}} </code> 

<h2> Plain text </h2>
<p> By default, most "math text" is in italics. To get plain Roman text,
you can use <code>\text{ }</code> or <code>\rm{ }</code>, depending on
whether you want spaces or math operations to be rendered or not. </p>
 \(\text{some text \frac 1 2}\) <code> \text{some text \frac 1 2} </code> \(\rm{some text \frac 1 2}\) <code> \rm{some text \frac 1 2} </code>
 \( g = 9.81~\rm{m/s^2}\)   <code> g = 9.81~\rm{m/s^2} </code> 
<p> The last example above uses a <code>~</code> (tilde) to put a blank space
between the number and the units. <code>\text{ }</code> doesn't work here
because the exponent wouldn't display properly. </p>
<h2> Parentheses, brackets, etc. </h2>
<p> For single-line expressions you can use parentheses and brackets
normally. However, braces are used by LaTeX itself for grouping
things, as you've seen above, so if you want them to actually appear, you
have to <b>escape</b> them by using backslashes. In this case the
braces lose their normal grouping function, so you have to use a second,
unescaped pair if you want them to be grouped. </p>
<p> For angle brackets, <code>\langle \rangle</code> looks better than
<code>&lt; &gt;</code>. </p>
  \(\{ a + b \}\)   <code> \{ a + b \} </code>  
   \(\frac 1 { \{ a + b \} }\)  <code> \frac 1 { \{ a + b \} } </code>  
  \(\langle x \rangle\)  <code> \langle x \rangle </code>  

<p> For "tall" expressions e.g. fractions, sums or integrals, you can make
parentheses etc. adjust their size automatically to fit their contents. </p>
 \(\left( \frac a b \right) \) <code> \left( \frac a b \right) </code>  
  \(\left[ \frac a b \right]\)  <code> \left[ \frac a b \right] </code>  
  \(\left\{ \frac a b \right\}\)  <code> \left\{ \frac a b \right\} </code>  
 \(\left| \frac a b \right|\)  <code> \left| \frac a b \right| </code>  
 \(\left\| \frac a b \right\|\)   <code> \left\| \frac a b \right\| </code>  
 \(\left&lt; \frac a b \right&gt;\)   <code> \left&lt; \frac a b \right&gt; </code>  
 \(\left[ 1 - \left( \frac v c \right)^2 \right]^{-1/2}\)   <code> \left[ 1 - \left( \frac v c \right)^2 \right]^{-1/2} </code>

<p> You can omit half of a pair of these "stretchy" delimiters by
replacing it with <code>\left.</code> or <code>\right.</code> as
appropriate. See the <b>Integrals</b> section below for an example. </p>
<h2> Vectors (physics) </h2>
<p> Different people and textbooks use different notations for vectors. I hope
yours is included in the following list. Also included are some common vector
math operators. Of course, you can use any of the vector notations with these
operators. I showed only one for brevity. </p>
  Vector 
 \(\mathbf A, \vec A, \tilde A\) 
 <code> \mathbf A, \vec A, \tilde A </code>  
  Unit vector  \(\hat {\mathbf A}, \hat A\)    <code> \hat {\mathbf A}, \hat A </code>  
  Scalar (dot) product  \(\vec A \cdot \vec B\)  <code> \vec A \cdot \vec B </code>  
  Vector (cross) product  \(\vec A \times \vec B\)   <code> \vec A \times \vec B </code>  
  Magnitude  \(| \vec A | \)  <code> | \vec A | </code>  

<h2> Derivatives </h2>
 \(y', y'', y^{(n)}, \dot x, \ddot x\)   <code> y', y'', y^{(n)}, \dot x, \ddot x </code>  
 \(\frac {df} {dx}, \frac {\partial f} {\partial x}\)   <code> \frac {df} {dx}, \frac {\partial f} {\partial x} </code> 
\(\frac {\partial^2 \psi} {\partial x^2}\)  <code> \frac {\partial^2 \psi} {\partial x^2} </code>  
 \(\nabla f, \nabla^2 \psi, \Box^2 \phi\)   <code> \nabla f, \nabla^2 \psi, \Box^2 \phi </code>  

<h2> Integrals </h2>
<p> With integrals, use <code>\,</code> to insert a bit of space (not as much
as a full space) before the  (or whatever the variable of integration
is) at the end. </p>
<p> In the second example, note how the vertical bar representing the evaluation
of limits is specified. It's half of a left/right pair of "stretchy"
absolute-value bars. The left half is hidden by specifying it as
<code>\left.</code> instead of <code>\left|</code>. </p>

 \( \int x^2e^x \, dx\)   <code> \int x^2e^x \, dx </code>  
 \(\int_a^b x^2 \, dx = \left. \frac 1 3 x^3 \right|_a^b\)   <code> \int_a^b x^2 \, dx = \left. \frac 1 3 x^3 \right|_a^b </code> 
 \(\oint \vec E \cdot d \vec A\)   <code> \oint \vec E \cdot d \vec A </code>  
 \(\int_{y=c}^d \int_{x=a}^b e^{x+y} \, dx \, dy\)   <code> \int_{y=c}^d \int_{x=a}^b e^{x+y} \, dx \, dy </code>  
  \(\iint, \iint_S, \iiint, \iiint_V \)  <code> \iint, \iint_S, \iiint, \iiint_V </code> 

<h2> Sums, products and limits </h2>
 \( \sum_{n=0}^\infty a_n x^n   \)                       <code> \sum_{n=0}^\infty a_n x^n                        </code>  
 \( \prod_{n=0}^\infty a_n      \)                       <code> \prod_{n=0}^\infty a_n                           </code>  
 \( \lim_{n \rightarrow +\infty} {\frac {\sin(x)} x} \)   <code> \lim_{n \rightarrow +\infty} {\frac {\sin(x)} x} </code> 

<h2> Greek letters </h2>
 Greek lowercase 
 \(\alpha \beta \gamma \delta \epsilon \zeta\)  <code>\alpha \beta \gamma \delta \epsilon \zeta</code>  
 \(\eta \theta \iota \kappa \lambda \mu\)   <code>\eta \theta \iota \kappa \lambda \mu</code>  
 \(\nu \omicron \xi \pi \rho \sigma \tau\)   <code>\nu \omicron \xi \pi \rho \sigma \tau</code>  
 \(\upsilon \phi \chi \psi \omega\)          <code>\upsilon \phi \chi \psi \omega       </code>  
 Greek lowercase (variant forms) 
\(\varepsilon \vartheta \varpi \varrho \varsigma \varphi\)  <code>\varepsilon \vartheta \varpi \varrho \varsigma \varphi</code> 
 Greek uppercase 
\(\Gamma \Delta \Theta \Lambda \Xi \Pi\) <code>\Gamma \Delta \Theta \Lambda \Xi \Pi</code>  
\(\Sigma \Upsilon \Phi \Psi \Omega\)  <code>\Sigma \Upsilon \Phi \Psi \Omega</code>  

<h2> Other special symbols </h2>
 Planck's constant \(\hbar\) <code>\hbar</code>  
 Infinity \(\infty\) <code>\infty</code>  
 Plus or minus \(\pm \mp\) <code>\pm \mp</code>  
 Parallel, perpendicular \(\parallel \perp\) <code>\parallel \perp</code>  
 For all \(\forall\)<code>\forall</code>  
 There exists \(\exists\)<code>\exists</code>  
  Ellipsis  \(\dots \cdots \vdots \ddots\)  <code> \dots \cdots \vdots \ddots </code>  
 <td colspan="4"> Arrows 
 \(\rightarrow \leftarrow \leftrightarrow  \)  <code> \rightarrow \leftarrow \leftrightarrow </code>  
 \(\Rightarrow \Leftarrow \Leftrightarrow  \) <code> \Rightarrow \Leftarrow \Leftrightarrow </code>  
 \(\mapsto \to                             \) <code> \mapsto \to                            </code>  
 \(\uparrow \downarrow                     \) <code> \uparrow \downarrow                    </code>  
 Relations 
  \(\gt \geq \lt \leq \neq\)      <code> \gt \geq \lt \leq \neq </code>  
  \(\gg \ll \equiv \)            <code> \gg \ll \equiv         </code>  
  \(\approx \cong \sim \)       <code> \approx \cong \sim     </code>  
  Absolute value, norm   \(|x|, \|x\| \) <code> |x|, \|x\| </code>  
  Complex conjugate, adjoint \( z^*, \bar z, z^\dagger \)  <code> z^*, \bar z, z^\dagger </code>  
  Dirac bracket notation \(\langle \phi | \psi \rangle\)   <code> \langle \phi | \psi \rangle </code>  
  Direct sum \(V \oplus W\) <code> V \oplus W \)</code>  
  Tensor product \( S \otimes T \) <code> S \otimes T </code>  
  Function composition \( f \circ g  \) <code>f \circ g </code>  
  Element of, not in \( x \in A, x \not\in A \) <code> x \in A, x \not\in A </code>  
  Subset \( A \subset B, B \subseteq C \) <code> A \subset B, B \subseteq C </code>  
  Set union, intersection \( A \cup B, A \cap B \) <code> A \cup B, A \cap B </code>  
  Multiple set union, intersection \( \bigcup_i A_i, \bigcap_i A_i \) <code> \bigcup_i A_i, \bigcap_i A_i </code>  
  Set subtraction \( A \setminus B \) <code> A \setminus B </code>  
  Binomial coefficient \( \binom n k, {_nC_r} \) <code> \binom n k, {_nC_r} </code>  

<h2> Special functions </h2>
<h2> Other special symbols </h2>
  Trigonometric \(\sin \cos \tan \sec \csc \cot \)  <code> \sin \cos \tan \sec \csc \cot </code>  
 <td > Inverse trig  \(\arcsin \arccos \arctan\)  <code> \arcsin \arccos \arctan </code>  
  \(\sin^{-1} \cos^{-1} \tan^{-1}\sec^{-1} \csc^{-1} \cot^{-1}\)  <code> \sin^{-1} \cos^{-1} \tan^{-1}\sec^{-1} \csc^{-1} \cot^{-1}</code>  
  Logarithmic \( \ln \log \log_2 \log_{10}\)   <code> \ln \log \log_2 \log_{10} </code>  
  Hyperbolic trig \(\sinh \cosh \tanh \coth \) <code> \sinh \cosh \tanh \coth</code>  
  Inverse hyperbolic trig \( \sinh^{-1} \cosh^{-1} \tanh^{-1} \coth^{-1}\) <code> \sinh^{-1} \cosh^{-1} \tanh^{-1} \coth^{-1} </code>  
  Other functions 
 \(\operatorname {arcsec}\)  <code> \operatorname {arcsec} </code>  

<p> Functions that are not supported directly, e.g. <code>arcsec arccsc arccot sech csch arcsinh arccosh arctanh
arcsech arccsch arccoth</code>, can be used via <code>\operatorname</code> as
shown in the last example above. </p>
<h2> Multiple lines </h2>
<p> To split an equation or expression into multiple lines, use <code>\\</code>.
</p>
<p> To align them in a specific way (e.g. at <code>=</code> signs), enclose them
between <code>\begin{align}</code> and <code>\end{align}</code>, and use
<code>&amp;</code> to specify the points at which they are to be aligned.
By default, this adds equation numbers at the right side of the page. To
suppress them, put <code>\nonumber</code> at the end of each line. </p>

  \( y = (a + b)^2 \\ = a^2 + 2ab + b^2 \)   <code> y = (a + b)^2 \\ = a^2 + 2ab + b^2 </code>  
  \(\begin{align}y &amp; = (a + b)^2 \\&amp; = a^2 + 2ab + b^2\end{align}\) <code> \begin{align}y &amp; = (a + b)^2 \\&amp; = a^2 + 2ab + b^2\end{align}</code>  
  \(\begin{align}y &amp; = (a + b)^2 \nonumber \\&amp; = a^2 + 2ab + b^2 \nonumber\end{align} \) 
      <code> \begin{align}y &amp; = (a + b)^2 \nonumber \\&amp; = a^2 + 2ab + b^2 \nonumber\end{align}</code>  
 <p> This sort of construction with <code>\begin{}</code> and <code>\end{}</code>
is called an <b>environment</b>. There are other environments for different kinds
of formatting. </p>

<h2> Matrices and Tables </h2>
 
\( \begin{pmatrix}1 &amp; 2 &amp; 3 &amp; 4 \\a &amp; b &amp; c &amp; d \\x &amp; y &amp; z &amp; w\end{pmatrix} \)
<code>\begin{pmatrix}1 &amp; 2 &amp; 3 &amp; 4 \\a &amp; b &amp; c &amp; d \\x &amp; y &amp; z &amp; w\end{pmatrix} </code>  

\(\begin{vmatrix}1 &amp; 2 &amp; 3 &amp; 4 \\a &amp; b &amp; c &amp; d \\x &amp; y &amp; z &amp; w\end{vmatrix}\)
 <code>\begin{vmatrix}1 &amp; 2 &amp; 3 &amp; 4 \\a &amp; b &amp; c &amp; d \\x &amp; y &amp; z &amp; w\end{vmatrix} </code>

\(\begin{bmatrix}1 &amp; 2 &amp; 3 &amp; 4 \\a &amp; b &amp; c &amp; d \\
x &amp; y &amp; z &amp; w\end{bmatrix}\) 
 <code> \begin{bmatrix}1 &amp; 2 &amp; 3 &amp; 4 \\a &amp; b &amp; c &amp; d \\
x &amp; y &amp; z &amp; w\end{bmatrix} </code>  

\(\begin{matrix}1 &amp; 2 &amp; 3 &amp; 4 \\a &amp; b &amp; c &amp; d \\
x &amp; y &amp; z &amp; w\end{matrix} \)
 <code> \begin{matrix}1 &amp; 2 &amp; 3 &amp; 4 \\a &amp; b &amp; c &amp; d \\
x &amp; y &amp; z &amp; w\end{matrix} </code>  
 
\(\begin{array}{|c|c|c|c|} 
\hline 1 &amp; 2 &amp; 3 &amp; 4 \\ 
\hline a &amp; b &amp; c &amp; d \\ 
\hline x &amp; y &amp; z &amp; w \\ 
\hline 
\end{array} \)
 <code> \begin{array}{|c|c|c|c|} 
\hline 1 &amp; 2 &amp; 3 &amp; 4 \\ 
\hline a &amp; b &amp; c &amp; d \\ 
\hline x &amp; y &amp; z &amp; w \\ 
\hline 
\end{array} </code>  

<h2> Cases </h2>
<p> The <code>cases</code> environment lets you specify functions piecewise, with
a left brace grouping the cases together. </p>

\(u(x) =
\begin{cases}
e^x &amp; \text{if } x \geq 0 \\
1 &amp; \text{if } x &lt; 0
\end{cases}  \)
 <code> u(x) =
\begin{cases}
e^x &amp; \text{if } x \geq 0 \\
1 &amp; \text{if } x &lt; 0
\end{cases} </code>  
 \(V(x) = \begin{cases}
0, &amp; x \lt 0 \\
V_0, &amp; 0 \leq x \lt L \\
0, &amp; x \geq L
\end{cases}\)   
 <code> V(x) = \begin{cases}
0, &amp; x \lt 0 \\
V_0, &amp; 0 \leq x \lt L \\
0, &amp; x \geq L
\end{cases} </code>  

<h2> Fonts </h2>

  Blackboard bold  \(\mathbb {ABCDE}\)    <code> \mathbb {ABCDE} </code>  
  Calligraphic \( \mathcal {ABCDE} \)  <code> \mathcal {ABCDE} </code>  
  Fraktur \( \mathfrak {ABCDE} \)  <code> \mathfrak {ABCDE} </code>  
  Sans serif  \( \mathsf {ABCDE} \)  <code> \mathsf {ABCDE} </code>  
  Roman (upright) \( \mathrm {ABCDE} \)  <code> \mathrm {ABCDE} </code>  
  Bold  \( \mathbf {ABCDE} \)  <code> \mathbf {ABCDE} </code>  
  Typewriter \( \mathtt {ABCDE} \) <code> \mathtt {ABCDE} </code>  

<h2>FAQ</h2>
<b>How do I write a fraction?</b>
<code>\(\frac{num}{denom}\) (inline)</code>

Examples: <code>\(\frac 1 3## and ##\frac{x + 2}{2x + 6}\)</code>

These render as 
Braces are necessary only if the numerator or denominator is longer than a single character.
<b>How do I write exponents and subscripts?</b>
<code>$$x^{-1}$$ and $$c_{10}(x + 1)^3$$</code>

These render as 
$$x^{-1}$$ and $$c_{10}(x + 1)^3$$
Braces are required for exponents and subscripts only if the exponent/subscript is longer than a single character.

[B]How do I write an integral?[/B]
<code>$$\int 2x + 3 ~dx$$ and $$\int_{t=0}^{10}t^2 + 1 ~dt$$</code>

These render as $$\int 2x + 3 ~dx$$ and $$\int_{t=0}^{10}t^2 + 1 ~dt

<b>How do I write a summation?</b>
<code>$$\sum_{k=0}^\infty a_k$$</code>

This renders as $$\sum_{k=0}^\infty a_k$$

Using $ delimiters, this renders as s $$\sum_{k=0}^\infty a_k$$
[B]How do I write Greek letters?[/B]
Here are a few lowercase Greek letters - <code>$$\alpha, \beta, \theta, \lambda, \sigma$$</code>
And a few uppercase Greek letters - <code>$$\Gamma, \Psi, \Phi, \Theta, \Lambda, \Sigma$$</code>

These render as $$\alpha, \beta, \theta, \lambda, \sigma$$ 
$$\Gamma, \Psi, \Phi, \Theta, \Lambda, \Sigma$$

                          
